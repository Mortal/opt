\documentclass[article,oneside,11pt]{memoir}
\usepackage[utf8]{inputenc}
\usepackage[sc]{mathpazo}
\usepackage{amsmath,amssymb}
\usepackage[final]{microtype}
\usepackage[danish]{babel}
\usepackage{enumitem}
\newsubfloat{figure}
\pagestyle{chapter}
\begin{document}
\begin{centering}
{\LARGE Genhusningsudvalget}

\vspace{1em} {\large
Hans Henrik Juhl

Helene Aagaard

Martin Christensen

Mathias Rav

My Maya Tvarnø

}
\end{centering}

\chapter{Optimering efter prioriteter}

\section{Fordele og ulemper}
\begin{minipage}[t]{0.45\linewidth}
\small
\subsection{Fordele}
\begin{itemize}[leftmargin=0ex]
\item[] Tilstræbt minimeret tilfældighed
\item[] Maksimering af medbestemmelse
\item[] Tilgodeser individets ønsker
\item[] Internetbaseret; kræver ikke fysisk gangmøde
\end{itemize}
\end{minipage}
\hspace{0.1\linewidth}
\begin{minipage}[t]{0.45\linewidth}
\small
\subsection{Ulemper}
\begin{itemize}[leftmargin=0ex]
\item[] Tilfældighed
\item[] Nogle folk får ikke deres vilje
\item[] Tager lang tid
\end{itemize}
\end{minipage}

\section{Model}

En beboer, der skal omhuses, kan give op til tre ønsker. For det første kan man
ønske, hvem man vil bo på samme gang med. Man kan i den sammenhæng kun ønske
folk, der skal genhuses i samme omgang. For det andet kan man ønske, om man vil
bo ved Langelandsgade (Koll. 4-5-6) eller ved Nørrebrogade (Koll. 7-8-9). For
det tredje kan man ønske, om man vil bo i stuen eller på 1./2./3. sal.

Derudover kan man give udtryk for, hvor vigtige de tre ting er i forhold til
hinanden. Man kan sige, at det kun betyder noget, hvem man kommer til at bo
med, eller at dette overhovedet ikke betyder noget. Det gør man ved at give
hver af tre ønsker vægtninger der tilsammen skal give 12. Det giver nemlig
mulighed for mange fordelinger, eksempelvis: 12-0-0, 8-4-0, 6-6-0, 6-3-3,
5-5-2, 4-4-4, 5-4-3.

\begin{figure}[h]
\begin{center}
\small
\begin{tabular}{ll}
\toprule
Beboer & 154 \\
\midrule
Personer: & 123 \\
vægt: 8 & 213 \\
& 321 \\
\midrule
Side: & $\text{\rlap{$\checkmark$}}\square$ Kollegium 4-5-6 \\
vægt: 4 & $\square$ Kollegium 7-8-9 \\
\midrule
Højde: & $\square$ Stue \\
vægt: 0 & $\square$ Ikke stue \\
\bottomrule
\end{tabular}

\vspace{1em}
\begin{minipage}[t]{0.6\textwidth}
\caption{En beboers ønsker. Beboeren er lige glad med stue/ikke stue, så denne
har fået vægt 0, og ingen af ønskerne er markeret.}
\end{minipage}
\end{center}
\end{figure}

\section{Praksis}

Beboerne, der skal omhuses, kan hver få tildelt en personlig adgangskode, som
de kan bruge til at logge ind på en hjemmeside, hvor ønskerne kan indtastes og
ændres. Ønskerne kan gemmes i en database, hvorfra vi kan hente inputdata til
programmet, der beregner løsningen. Folk skal have omtrent en uge til at taste
ønsker ind, og vi bruger formodentlig et par dage på at vente på at løsningen
beregnes.

\section{Beregning}

Vi skal nu finde ud af, hvordan én tildeling af beboere til gange er bedre end
en anden. Det gør vi ved at indføre begrebet \emph{godhed}, som er et kommatal
der angiver hvor god en tildeling er. Den værste godhed, en tildeling kan have,
er $0.0$, hvis ingen beboere har fået det, som de vil have det. Den bedste
godhed er lig antallet af beboere, der skal omhuses, ganget med 12 (hvis $51$
skal omhuses er den bedste godhed $612$).

For at beregne godhed skriver vi først op for hver beboer deres værelsesnummer
samt hvilken gang de får lov at bo på. Ved hver beboer skriver vi så på linjen,
om han/hun er kommet på samme gang som nogen af de personer, han/hun ønskede: 1
eller 0 for hhv. ja og nej. Så skriver vi ved siden af dette, om beboeren på
den side af unisøen, som han/hun ønskede: 1 eller 0.  Derefter skriver vi, om
ønsket om etagen er opfyldt: 1 eller 0. Disse tre tal kalder vi for hhv. $g_p$,
$g_s$ og $g_e$.  Beboerens vægtning af disse tre forhold kalder vi tilsvarende
hhv. $v_p$, $v_s$ og $v_e$.  Nu beregner vi, hvor godt det ser ud for beboeren.
Den individuelle godhed er
\[G = v_p\cdot g_p + v_s\cdot g_s + v_e\cdot g_e.\]

Det viser sig, at tallet $G$ altid ligger mellem 0 og 12, hvis
summen af vægtningerne er 12.  Den samlede godhed for tildelingen er så summen
af alle disse $G$ for hver beboer.

\begin{figure}[h]
\begin{center}
\caption{To forskellige tildelinger med de samme fire beboere.}
\subtop[Eksempel på tildeling med samlet godhed 16]{
\begin{tabular}{ccccccccc}
\toprule
Beboer & Gang & $g_p$ & $g_s$ & $g_e$ & $v_p$ & $v_s$ & $v_e$ & $G$ \\ \midrule
123 & 4-St & 1 & 0 & 0 & 4 & 8 & 0 & 4  \\ \midrule
132 & 4-St & 1 & 1 & 0 & 6 & 6 & 0 & 12 \\ \midrule
41  & 7-2  & 0 & 1 & 1 & 8 & 2 & 0 & 2  \\ \midrule
8   & 8-2  & 0 & 0 & 1 & 6 & 6 & 0 & 0  \\ \midrule
Sum &      &   &   &   &   &   &   & 16 \\ \bottomrule
\end{tabular}
}

\subtop[Eksempel på tildeling med samlet godhed 28]{
\begin{tabular}{ccccccccc}
\toprule
Beboer&Gang  &$g_p$&$g_s$&$g_e$&$v_p$&$v_s$&$v_e$&$G$ \\ \midrule
123   & 7-St & 0   & 1   & 0   & 4   & 8   & 0   & 8  \\ \midrule
132   & 8-1  & 0   & 0   & 1   & 6   & 6   & 0   & 0  \\ \midrule
41    & 4-3  & 1   & 0   & 1   & 8   & 2   & 0   & 8  \\ \midrule
8     & 4-3  & 1   & 1   & 1   & 6   & 6   & 0   & 12 \\ \midrule
Sum   &      &     &     &     &     &     &     & 28 \\ \bottomrule
\end{tabular}
}
\end{center}
\end{figure}

Nu skriver vi et computerprogram, der forsøger med mange milliarder forskellige
tildelinger og husker på den med den højeste godhed. Med omtrent 50 beboere og
24 gange er der så mange forskellige måder, man kan tildele beboere til gange,
at vi aldrig kan nå at prøve alle muligheder. Vi kan dog skrive et program,
hvor vi forventer at finde en af de bedste tildelinger på overkommelig tid.

\end{document}
